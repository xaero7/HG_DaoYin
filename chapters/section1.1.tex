\chapter{必修1 $\cdot$ 数据与计算}\index{数据与计算}
\section{数据与信息}\index{数据与信息}

\begin{enumerate}
%% ============= 1
\item 答案:D。考查数据与信息概念的理解,数据、数字的差别。
	\begin{enumerate}[label=\Alph*.]
	\item 数据是对客观事物的符号表示,如图形符号、数字、字母等。在计算机中的表示形式可以是文字、图形、图像、音频、视频等。
	\item 数据可以加过处理,但显然你可以让他失去原有的价值。
	\item 数字放到特定的环境、语境下才有意义,即要有上下文才有含义。
	\item 正确。
	\end{enumerate}

%% ============= 2
\item 答案:B。考查信息概念的理解,信息的特征。
	\begin{enumerate}[label=\Alph*.]
	\item 实验误差是测量值和真实值之间的偏差,不是虚假信息。
	\item 正确。
	\item 同一个信息对于不同的人价值可能不一样。
	\item 信息是信号、消息中所包含的含义,必须依附与数字、文字、图形、图像等载体。
	\end{enumerate}

%% ============= 3
\item 答案:D。考查信息概念的理解,信息的特征。
	\begin{enumerate}[label=\Alph*.]
	\item 互联网上只有已数字化的信息,没有数字化当然查不到。
	\item 知识的获得是人利用自身已有的知识对信息进行加工,进而将新的信息纳入自己的知识结构的过程。检索到也只是看到,并不一定已内化成自己的知识。
	\item 天才也要记单词啊。
	\item 正确。
	\end{enumerate}

%% ============= 4
\item 答案:A。
	\begin{enumerate}[label=\Alph*.]
	\item 。
	\item 。
	\item 。
	\item 。
	\end{enumerate}

%% ============= 5
\item 答案:C。
	\begin{enumerate}[label=\Alph*.]
	\item 若化成十进制计算:$10H=16D, 10B = 2D, 16D+2D=18D$。
	\item $1AH + 2AH = 44H$,注意十六进制下$A + A$等于$14$。
	\item 正确。
	\item 若化成十六进制计算:$10D + 10B = AH + 2H = CH$,即十六进制的值是$C$。
	\end{enumerate}

%% ============= 6
\item 答案:D。
	\begin{enumerate}[label=\Alph*.]
	\item 。
	\item 。
	\item 。
	\item 。
	\end{enumerate}

%% ============= 7
\item 答案:D。
	\begin{enumerate}[label=\Alph*.]
	\item 。
	\item 。
	\item 。
	\item 。
	\end{enumerate}

%% ============= 8
\item 答案:D。
	\begin{enumerate}[label=\Alph*.]
	\item 。
	\item 。
	\item 。
	\item 。
	\end{enumerate}

%% ============= 9
\item 答案:B。
	\begin{enumerate}[label=\Alph*.]
	\item 。
	\item 。
	\item 。
	\item 。
	\end{enumerate}

%% ============= 10
\item 答案:B。
	\begin{enumerate}[label=\Alph*.]
	\item 。
	\item 。
	\item 。
	\item 。
	\end{enumerate}

%% ============= 11
\item 考查信息编码、容量计算。
	\begin{enumerate}[label=$(\arabic*)$]
	\item {\kaishu 视频容量=每帧图像容量$\times$帧频,每帧图像容量=像素点数$\times$量化位数}。依题意,单张图像容量是:$\dfrac{1280 \times 720 \times 24}{8 \times 1024 \times 1024} \approx 2.64$MB。因此视频容量是:$2.64 \times 5 \times 60 \times 25 = 19800$MB。压缩比至少是$39.6:1$才能压缩到500MB以内。答案是$40:1$。
	\item 加入数据不会改变原来的压缩比,相当于不会改变原先的压缩编码方式(真因为如此,加入音频后的视频容量增加,压缩比不变的话,压缩之后的作品容量也增加,势必会超过500MB,因此必须重新设定新的、更大的压缩比才能压缩到500MB以内,这应该是本题想考查的一个实际应用情景)。
	\item 压缩会使画面不清晰,原因压缩比太大,或者压缩算法太差。改进办法是可以换一种压缩算法(换一个压缩软件),或者保证内容完整的前提下,减少画面尺寸、缩短时长等。
	\end{enumerate}

\end{enumerate}


\newpage
