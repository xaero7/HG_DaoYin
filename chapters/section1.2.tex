\section{算法与程序设计}\index{算法与程序设计}

\begin{enumerate}
%% ============= 1
\item 答案:A。
	\begin{enumerate}[label=\Alph*.]
	\item 。
	\item 。
	\item 。
	\item 。
	\end{enumerate}

%% ============= 2
\item 答案:B。考查信息概念的理解,信息的特征。
	\begin{enumerate}[label=\Alph*.]
	\item 实验误差是测量值和真实值之间的偏差,不是虚假信息。
	\item 正确。
	\item 同一个信息对于不同的人价值可能不一样。
	\item 信息是信号、消息中所包含的含义,必须依附与数字、文字、图形、图像等载体。
	\end{enumerate}

%% ============= 3
\item 答案:D。考查信息概念的理解,信息的特征。
	\begin{enumerate}[label=\Alph*.]
	\item 互联网上只有已数字化的信息,没有数字化当然查不到。
	\item 知识的获得是人利用自身已有的知识对信息进行加工,进而将新的信息纳入自己的知识结构的过程。检索到也只是看到,并不一定已内化成自己的知识。
	\item 天才也要记单词啊。
	\item 正确。
	\end{enumerate}

%% ============= 4
\item 答案:A。
	\begin{enumerate}[label=\Alph*.]
	\item 。
	\item 。
	\item 。
	\item 。
	\end{enumerate}

%% ============= 5
\item 答案:C。
	\begin{enumerate}[label=\Alph*.]
	\item 若化成十进制计算:$10H=16D, 10B = 2D, 16D+2D=18D$。
	\item $1AH + 2AH = 44H$,注意十六进制下$A + A$等于$14$。
	\item 正确。
	\item 若化成十六进制计算:$10D + 10B = AH + 2H = CH$,即十六进制的值是$C$。
	\end{enumerate}

%% ============= 6
\item 答案:D。
	\begin{enumerate}[label=\Alph*.]
	\item 。
	\item 。
	\item 。
	\item 。
	\end{enumerate}

%% ============= 7
\item 答案:D。
	\begin{enumerate}[label=\Alph*.]
	\item 。
	\item 。
	\item 。
	\item 。
	\end{enumerate}

%% ============= 8
\item 答案:D。
	\begin{enumerate}[label=\Alph*.]
	\item 。
	\item 。
	\item 。
	\item 。
	\end{enumerate}

%% ============= 9
\item 答案:B。
	\begin{enumerate}[label=\Alph*.]
	\item 。
	\item 。
	\item 。
	\item 。
	\end{enumerate}

%% ============= 10
\item 答案:B。
	\begin{enumerate}[label=\Alph*.]
	\item 。
	\item 。
	\item 。
	\item 。
	\end{enumerate}

%% ============= 11
\item 答案:B。
	\begin{enumerate}[label=\Alph*.]
	\item 。
	\item 。
	\item 。
	\item 。
	\end{enumerate}

%% ============= 12
\item 答案:B。
	\begin{enumerate}[label=\Alph*.]
	\item 。
	\item 。
	\item 。
	\item 。
	\end{enumerate}


%% ============= 13
\item 答案:A。考查Python循环语句、双重循环程序的阅读理解。
\columnratio{0.46}
\begin{paracol}{2}
\begin{lstlisting}[numbers=left]
for i in range(1, 7):
    for j in range(1, 7):
        if j <= i:
            print(j, end=" ")
        else:
            print("", end="")
    print()
\end{lstlisting}
\switchcolumn
固定第$1$行处的外层循环$i$的值为$1$时,内层循环$j$从$1$变化到$6$,对于每一个$j$,当$j \le i$时输出$j$的值,否则输出空值。因此当$i=1$时,输出$1$,然后换行;当$i=2$时,输出$1 \; 2$然后换行,当$i=3$时输出$1 \; 2 \; 3$然后换行……,答案选A。
\end{paracol}

%% ============= 14
\item 答案:

%% ============= 15
\item 答案:

%% ============= 16
\item 答案:

%% ============= 17
\item 答案:

%% ============= 18
\item 答案:

%% ============= 19
\item 答案:

%% ============= 20
\item 考查应用Python程序解决实际问题的能力。考查字符串的处理与应用。
	\begin{enumerate}[label=$(\arabic*)$]
	\item 考查题意的理解,这是理解题目情景的关键。“we put the bed in the bedroom”中有两处“bed”,会被替换两次。	
	\item 阅读与推导过程:
\setcounter{qnumber}{1}
\begin{lstlisting}[numbers=left]
text = input("输入原文字符串: ")
key = input("输入要查找的字符串: ")
rs = input("输入替换字符串:")
result = "";  count = 0;  i = 0;  n = len(text)
while i < n - len(key) + 1:
 s = text[`\clozeblank{2}`]
 if key == s:
     result += rs
     count += 1
     i += len(key)
 else:
     result += text[i]
     i += 1
`\clozeblank{2}`
if count > 0:
 print("替换的次数: ", count)
 print("替换后的结果: ", result)
else:
 print("要查找的内容不存在")
\end{lstlisting}
		\begin{enumerate}[label=$(\alph*)$]
		\item 第5行的循环和$n$有关,而$n$是原文的长度,因此第5行的循环是在扫描原文的每个字符。
		\item 从$i$的变化上看,当第8行两个字符相等时,$i$往后移动与key一样的长度;当两个字符不等时,$i$往后移动1个字符长度,所以$i$是指示了原文text中待比较字符串的索引位置信息。
		\item 循环中第7行判定了key是否与$s$相等,那么$s$就需要从原文中截取一个字符串,再与key作相等比较,因此第①空应该填写原文字符串的切片,切片的起始值是当前$i$的值,切片的长度应该与key的长度相等,于是第①空答案是\lstinline|i:i+len(key)|。
		\item 在解题时一定要用样例带入后阅读,比如原文\lstinline|text="Abedrbedom"|,待替换字符串\lstinline|key="bed"|,那么当$i=1,5$时分别找到两处“bed”,如下图所示。当$i$指向8号位置时,剩余字符串已不足3位(即待查找值key的长度),也就无需继续循环,这也是第5行while循环条件是\lstinline|i < n - len(key) + 1|而不是\lstinline|i < n|的原因。但是这样带来的后果是剩余的字符串无法原样连接到result中取,如下图中的最后两个字符“om”。因此需要在循环结束时,第14处将剩余字符串连接到最终结果上。第②空的答案是\lstinline|result += text[i:]|,其中切片的终止端点写明$n$亦可。
\begin{lstlisting}
0 1 2 3 4 5 6 7 8 9
A b e d r b e d o m
  ↑
  i=1时找到第一处,字符串替换后,i=i+3,指向4号位置
----------------------------------------------------------------------
0 1 2 3 4 5 6 7 8 9
A b e d r b e d o m
          ↑
          i=5时找到第二处,字符串替换后,i=i+3,指向8号位置
\end{lstlisting}
		\end{enumerate}
	

	\end{enumerate}

\end{enumerate}



