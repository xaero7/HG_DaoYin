\section{算法与程序设计}\index{算法与程序设计}

\begin{enumerate}
%% ============= 1
\item 答案:A。

%% ============= 2
\item 答案:A。

%% ============= 3
\item 答案:D。

%% ============= 4
\item 答案:C。

%% ============= 5
\item 答案:B。

%% ============= 6
\item 答案:C。选项A、B都可以用解析法,用公式计算。

%% ============= 7
\item 答案:D。使用列表法模拟流程图的执行。

%% ============= 8
\item 答案:A。

%% ============= 9
\item 答案:D。

%% ============= 10
\item 答案:C。计算机表达式:写在同一行,没有上、小标的数学写法,一般只有字母、数字等标识符,$\pi$不是英文字母。

%% ============= 11
\item 答案:B。

%% ============= 12
\item 答案:B。

%% ============= 13
\item 答案:A。考查Python循环语句、双重循环程序的阅读理解。
\columnratio{0.46}
\begin{paracol}{2}
\begin{lstlisting}[numbers=left]
for i in range(1, 7):
    for j in range(1, 7):
        if j <= i:
            print(j, end=" ")
        else:
            print("", end="")
    print()
\end{lstlisting}
\switchcolumn
固定第$1$行处的外层循环$i$的值为$1$时,内层循环$j$从$1$变化到$6$,对于每一个$j$,当$j \le i$时输出$j$的值,否则输出空值。因此当$i=1$时,输出$1$,然后换行;当$i=2$时,输出$1 \; 2$然后换行,当$i=3$时输出$1 \; 2 \; 3$然后换行……,答案选A。
\end{paracol}

%% ============= 14
\item 答案:(1)确定性,让输入错误次数加一。(2)②

%% ============= 15
\item 答案:(1)$|s * s - 2| \ge 0.0001?$,算法中可以用数学符号,也可以用计算机表达式。(2)B。

%% ============= 16
\item 答案:(1)①②⑤ (2)①顺序结构 ②$i=i+m$

%% ============= 17
\item 答案:① w <= 20 ② 0.6*15+5*1.4+(w-20)*2.1

%% ============= 18
\item 答案:① a=b ②str(c)或str(b)

%% ============= 19
\item 答案:① n=len(h) ② code[a]*16**(n-i-1)

%% ============= 20
\item 考查应用Python程序解决实际问题的能力。考查字符串的处理与应用。
	\begin{enumerate}[label=$(\arabic*)$]
	\item 考查题意的理解,这是理解题目情景的关键。“we put the bed in the bedroom”中有两处“bed”,会被替换两次。	
	\item 阅读与推导过程:
\setcounter{qnumber}{1}
\begin{lstlisting}[numbers=left]
text = input("输入原文字符串: ")
key = input("输入要查找的字符串: ")
rs = input("输入替换字符串:")
result = "";  count = 0;  i = 0;  n = len(text)
while i < n - len(key) + 1:
 s = text[`\clozeblank{2}`]
 if key == s:
     result += rs
     count += 1
     i += len(key)
 else:
     result += text[i]
     i += 1
`\clozeblank{2}`
if count > 0:
 print("替换的次数: ", count)
 print("替换后的结果: ", result)
else:
 print("要查找的内容不存在")
\end{lstlisting}
		\begin{enumerate}[label=$(\alph*)$]
		\item 第5行的循环和$n$有关,而$n$是原文的长度,因此第5行的循环是在扫描原文的每个字符。
		\item 从$i$的变化上看,当第8行两个字符相等时,$i$往后移动与key一样的长度;当两个字符不等时,$i$往后移动1个字符长度,所以$i$是指示了原文text中待比较字符串的索引位置信息。
		\item 循环中第7行判定了key是否与$s$相等,那么$s$就需要从原文中截取一个字符串,再与key作相等比较,因此第①空应该填写原文字符串的切片,切片的起始值是当前$i$的值,切片的长度应该与key的长度相等,于是第①空答案是\lstinline|i:i+len(key)|。
		\item 在解题时一定要用样例带入后阅读,比如原文\lstinline|text="Abedrbedom"|,待替换字符串\lstinline|key="bed"|,那么当$i=1,5$时分别找到两处“bed”,如下图所示。当$i$指向8号位置时,剩余字符串已不足3位(即待查找值key的长度),也就无需继续循环,这也是第5行while循环条件是\lstinline|i < n - len(key) + 1|而不是\lstinline|i < n|的原因。但是这样带来的后果是剩余的字符串无法原样连接到result中取,如下图中的最后两个字符“om”。因此需要在循环结束时,第14处将剩余字符串连接到最终结果上。第②空的答案是\lstinline|result += text[i:]|,其中切片的终止端点写明$n$亦可。
\begin{lstlisting}
0 1 2 3 4 5 6 7 8 9
A b e d r b e d o m
  ↑
  i=1时找到第一处,字符串替换后,i=i+3,指向4号位置
----------------------------------------------------------------
0 1 2 3 4 5 6 7 8 9
A b e d r b e d o m
          ↑
          i=5时找到第二处,字符串替换后,i=i+3,指向8号位置
\end{lstlisting}
		\end{enumerate}
	\end{enumerate}

%% ============= 21
\item 考查应用Python程序解决实际问题的能力。考查字符串的处理与应用。
	\begin{enumerate}[label=$(\arabic*)$]
	\item “Good”中的四个字母来自键盘不同的两行按键,故答案是No。
	\item 阅读与推导过程:
\setcounter{qnumber}{1}
\begin{lstlisting}[numbers=left]
def to_lower(ch):       # 转小写字母
    if ch >= "A" and ch <= "Z":
        return chr(ord(ch) + 32)
    else:
        `\clozeblank{2}`
line_1 = "qwertyuiop"   # 键盘第一行字母
line_2 = "asdfghjkl"    # 键盘第二行字母
line_3 = "zxcvbnm"      # 键盘第三行字母
char = input()
c1 = 0
c2 = 0
c3 = 0
n = len(char)
for ch in char:
    ch = `\clozeblank{2}`
    if ch in line_1:
        c1 += 1
    elif ch in line_2:
        c2 += 2
    elif ch in line_3:
        c3 += 3
if c1 == n or c2 == n * 2 or c3 == n * 3:
    print("yes")
else:
    print("no")
\end{lstlisting}
	\begin{enumerate}[label=$(\alph*)$]
	\item 第①空容易填:第2行if语句判定了大写字母,第3行将其转成小写字母,因此非大写字母时直接返回ch,答案是\lstinline|return ch|。
	\item 第14行循环语句遍历提取了输入字符串char中的每个字符,在第16、18、20行分别判定了是否是键盘上哪一行的字母:第一行则$c1$加1,第二行则$c2$加2,第三行则$c3$加3,由此可以断定,在第②空处需要将字母规范化——统一转成小写字母,这就需要调用\lstinline|to_lower()|函数,因此答案是\lstinline|to_lower(ch)|,参数是当前扫描到的字符ch。
	\item 对于第22行条件的理解:如果输入字符char都来自键盘第一行,那么$c1$的值与char字符串长度相等,因此$c1$每次都加1;同理,如果都来自第二行,则$c2$的值是char字符串长度的两倍,因为$c2$每次都加2;$c3$的值亦同理。
	\end{enumerate}
	\item \lstinline|c1 += 1|的含义是第一行的字符数量增加1
	\end{enumerate}

%% ============= 22
\item 考查应用Python程序解决实际问题的能力。考查随机数函数、枚举算法。
	\begin{enumerate}[label=$(\alph*)$]
\setcounter{qnumber}{1}
\begin{lstlisting}[numbers=left]
import random
n = int(input("请输入要产生的英文字符串长度: "))
s = ""
for i in range(n):
	# randint(1,58):随机生成一个[1,58]范围内整数,字母A的ASCII码值为65
    s += chr(64 + random.randint(1, 58))  
print(s)
ans = input("请按样例输入: ")
c = 0
for i in range(n):
    if `\clozeblank{2}`:
        c += 1
p = `\clozeblank{2}`
print("正确数量: ", c, ",正确率为: ", p, "%")
\end{lstlisting}
	\item 判断两个字符串有多少个字符相同,可以用枚举算法:遍历每个字符串的每一位,分别判定是否相等。
	\item 由第10行的for循环语句知,$i$取遍了$[0,-1]$的每个数,这相当于字符串的索引值。而第11行处的条件成立时,变量$c$的值加1,又由第14行的输出可知$c$是正确单词的个数。因此第①空是判定两个字母是否相等,原始字符串是s,用户输入字符串是ans,因此答案是\lstinline|s[i] == ans[i]|。
	\item 变量$p$是什么?同样可以看14行的输出语句——$p$是正确率百分比。因此$p$的计算方式是正确个数除上总个数,答案可以是\lstinline|c / n * 100|。题意没有说如何保留小数,也没有输出示例,因此这个答案也可以。参考答案是\lstinline|int(c / n * 100 + 0.5)|,它的功能是四舍五入保留整数。
	\end{enumerate}


%% ============= 23
\item 考查应用Python程序解决实际问题的能力。考查进制转换解析算法、字符串应用。
	\begin{enumerate}[label=$(\alph*)$]
\setcounter{qnumber}{1}
\begin{lstlisting}[numbers=left]
def conv(s):
    ans = ""
    if s > "9":
        `\clozeblank{2}`
    else:
        s = int(s)
    while s > 0:
        k = s % 2
        s //= 2
        `\clozeblank{2}`
    for i in range(4 - len(ans)):
        ans = "0" + ans
    return ans
s = "2A08:CCD6:0088:108A:0011:0002:202F:AA05"
ans = ""
flag = False
for i in s:
    if i == ":":
        `\clozeblank{2}`
        ans += i
    elif i != "0" or flag == True :
        `\clozeblank{2}`
        flag = True
print("原IPv6地址为:", s)
print("去前导零后:", ans)
\end{lstlisting}
	\item 当自定义函数比较复杂时,可以从主程序开始阅读。那么从第14行开始阅读程序:第17行遍历取出了字符串$s$中的每个字符,$s$是个十进制模式的IPv6地址字符串,依题意需要将它转成二进制模式,可以猜测本题处理思路就是逐个取出$s$的每个字符并转换成二进制并输出结果。
	\item 第18行是$i$的值是冒号的情况,这意味着冒号前面一段IPv6已转换完,$i$中的值直接连接到最后结果字符串ans变量的后面,ans变量的功能也还是从最后一行的输出语句中得到。但是第③空还不知道填什么。
	\item 第21行是说当$i$不是0,或者flag的值是True时执行第22、23行程序。容易相当,当$i$非0时必然要转成二进制格式,因此这里需要调用前面的conv()函数,调用结果应该是$i$字符对应的二进制串,同样要把二进制串连接到ans变量后面,于是第④空的答案应该是\lstinline|ans += conv(i)|,其中$i$就是当前需要转换成二进制的十六进制字符。
	\item 从第21行的elif判定结果看,当$i \neq 0$时,flag的结果会变成True,一个隐含的情况是当$i=0$但flag值是True的时候,也会执行21、22两行代码,即也会将该“0”转成二进制串。结合题意“前导零可以省略”可知,非前导(中间的)零需要转换。由此断定flag的值为True表示当前有非前导的零(需要转换);flag的值为False时,若出现零则是前导的零。再结合第16行f初值False可知,这样的推论是合适的。
	\item 因此,第③空是出现冒号后,下一次得到的字符串若是零,该零必然是前导的零,于是这里填\lstinline|flag = False|。
	\item 转到conv()函数,可以断定参数变量$s$是待转换的十六进制字符串,这在第8、9两行的循环模2取余也可以得到验证(转二进制的方法就是除二取余法)。
	\item 但是$s$是十六进制,除二取余之前需要转成十进制。阅读第$1 \sim 6$行的if语句可以看出,若$s \le 9$,则直接取整(此时十六进制值与十进制相等);否则要把“A”转成10,把“B”转成11……把“F”转成15。把字母转整数可以用ASCII码函数ord(),本题答案是\lstinline|ord(s) - 65 + 10|。
	\item 第②空是将余数$k$连接成二进制串的语句,注意最先除二取余得到的余数是最低位,最后得到的余数是最高位,因此本题答案需要将$k$转成字符串后连接到答案变量ans的前面。本题填\lstinline|ans = str(k) + ans|。	
	\end{enumerate}


%% ============= 24
\item 考查应用Python程序解决实际问题的能力,考查列表的应用。
	\begin{enumerate}[label=$(\alph*)$]
\setcounter{qnumber}{1}
\begin{lstlisting}[numbers=left]
import random as rd
data = [180,283,385,170,276,384,180,285,380,190,295,390,170,272,372]
s = [0, 0, 0]                   # 存储3个作品的得分
ans = []                        # 存储并列最高平均分的作品号
maxb = 0
for i in range(len(data)):
    zp = `\clozeblank{2}`   # 分离出作品编号
    fs = data[i] % 100
    `\clozeblank{2}`        # 累加当前作品的得分
for j in range(3):
    `\clozeblank{2}`
    print("作品", j+1, "平均分为", s[j])
    if s[j] > maxb:
        maxb = s[j]
for z in range(3):              # 查找并列最高平均分
    if `\clozeblank{2}`:
        ans.append(z + 1)       # 将数据添加到列表ans尾部
print("平均分最高作品号是: ", ans)
\end{lstlisting}
	\item 由第6行的循环范围知,该for循环遍历了data列表的每个元素,$i$是其索引值。
	\item 分离字符串可以用切片,分离整数的各个数位可以用对十取余数,或者整除十。从循环中的第8行知,\lstinline|data[i] % 100|就是该整数的十位和各位上的数,即第$i$号数据中的得分值。第②空,分离百位数可以模仿这写\lstinline|data[i] // 100|,除以100之后的整数部分就是作品号。
	\item 按注释,第②空应是累加fs的值到列表$s$中的某个位置上,这个位置应该与作品编号有关。注意到第7行分离出来的作品编号都是从“1”开始数的,而$s$的索引值是从“0”开始计的,因此需要修正,答案是\lstinline|s[zp-1] += fs|。
	\item 因为有3个作品,因此第10行的3次循环应该遍历了每个作品,并输出了相应的平均分。也就是说,第12行输出的s[j]是第$j$号作品的平均分,而之前$s$中保存的是每个作品的总分,因此第③空需要求平均分,答案是\lstinline|s[j] /= 5|。
	\item 第13行是打擂算法:若当前平均分s[j]大于“擂台”上的数maxb,则让s[j]留在擂台上。所以maxb保存的是最大值。
	\item 第④空的条件成立时将$z$存入列表ans,而由最后一行的输出结合输出图示看,列表ans中存放了所有得分都是最高的作品编号。第④空就得填写某个作品均值与maxb相等时执行插入操作,答案是\lstinline|s[z] == maxb|。
	\end{enumerate}






































\end{enumerate}


\newpage
