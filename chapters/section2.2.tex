\section{信息系统的支撑技术}\index{支撑技术}

\begin{enumerate}
%% ============= 1
\item 答案:A。选项D是存储器的分类,选项C是内存的分类。

%% ============= 2
\item 答案:C。就是上一题选项C所描述的内容。

%% ============= 3
\item 答案:A。频率找$Hz$,CPU的主频一般是GHz。

%% ============= 4
\item 答案:C。选项A:内存是4GB,其中能用的只有3.46GB(另外0.54GB可能共享给显存使用了)。选项B:Windows 10就是一个系统软件。选项C:你看到这张截图说明有显示器,显示器就是输出设备。选项D:图上写着没有触控输入。

%% ============= 5
\item 答案:B。选项AC这两个只能在苹果操作系统macOS上安装。由图知,这个系统是在64位CPU上安装了32位的操作系统,所以其他应用软件都只能安装32位及以下的软件。64位相当于64个车道,同时能处理64位的二进制。不过它只安装了32位的操作系统,相当于最多同时只有32辆车在这车道上跑,因此它能安装的应用软件也只能是32位及以下的软件。

%% ============= 6
\item 答案:D。选项B:计算机与移动终端的硬件架构、原理都类似,但是操作系统不一样,所以应用软件大部分都无法通用。

%% ============= 7
\item 答案:B。显然你电脑的配置需要大于等于应用软件的配置才行,比较来看,小陈电脑显卡的显存才1GB,应用软件要求2GB,无法满足。

%% ============= 8
\item 答案:D。

%% ============= 9
\item 答案:C。

%% ============= 10
\item 答案:B。

%% ============= 11
\item 答案:D。考查网络的组成。选项A:教材P76提及“(计算机系统的)服务器负责数据处理和网络控制,是网络的主要资源”。应该说,网络组成的三大要素:计算机系统、数据通信系统、网络软件和网络协议都重要😀。选项B:数据通信系统包括传输介质和网络互联设备。选项C:网络软件包括服务器和客户端的网络操作系统、通信软件、管理和服务软件。

%% ============= 12
\item 答案:A。选项D:WWW是万维网的Web服务

%% ============= 13
\item 答案:A。

%% ============= 14
\item 答案:A。光纤猫的网线是接到无线路由器的WAN口上。

%% ============= 15
\item 答案:有线信号比无线信号更稳定;台式机的性能可能优于笔记本电脑。

%% ============= 16
\item 考查信息系统的组成,人工智能应用。
	\begin{enumerate}[label=$(\arabic*)$]
	\item 射频识别(RFID), ④,⑤,电子标签存储图书信息,可被读写器识别。
	\item 输入设备。
	\item 输出设备。
	\end{enumerate}

%% ============= 17
\item 答案如下:
	\begin{enumerate}[label=$(\arabic*)$]
	\item D。
	\item B。
	\item 答案:B。分支结构的程序请绘制数轴进行理解:选项A中$sy \ge 100$与$250 \le sy \le 400$有交集,当$sy=300$时,它只执行$sy \ge 100$成立的分支,即只亮1号灯。选项C:当$sy \ge 400$时亮了1号灯,当$250 \le sy \le 400$时亮了1、2号灯;当$sy < 250$时亮了1、2、3号灯,与题意不符。选项D:与A类似,第一个分支$sy \ge 100$范围太广了。
	\end{enumerate}

%% ============= 18
\item 考查信息系统搭建相关知识:
	\begin{enumerate}[label=$(\arabic*)$]
	\item 用浏览器访问网站,必然有HTTP协议(因为URL都是http打头的(仅限学选考知识范围)),另外现有的HTTP协议都是基于TCP/IP协议的,因此这里可以选填两种协议:HTTP和TCP/IP。
	\item 用浏览器访问必然是B/S架构,理由是它用的客户端软件是浏览器,不是专门定制的应用程序或APP。
\begin{lstlisting}[numbers=left]
# 导入Flask相关的库代码略
@app.route("/")
def index():
    # 查询最近上传的传感器数据记录,存入列表a,代码略
    return render_template("index.html", data=a)
@app.route("/input", methods=["GET", "POST"])
def add():
    ip = request.remote_addr
    tp = int(request.args.get("id"))
    value = int(request.args.get("val"))
    sj = time.strftime("%Y-%m-%d %H:%M", time.localtime(time.time()))
    # 以下代码用于连接数据库并处理环境数据
    conn = sqlite3.connect("mydata.db")
    cur = conn.cursor()
    sql = f"INSERT INTO jclog (type, value, upload_time, ip) VALUES ({tp}, {value}, '{sj}', '{ip}')"
    cur.execute(sql)
    conn.commit()
    return redirect(url_for("index"))
if __name__ == "__main__":
    app.run(debug=True, host="0.0.0.0", port=5000)
\end{lstlisting}
	\item 智能终端的代码主要是读取传感器数据,向Web服务发送数据,执行服务器返回的控制执行器的命令。Flask模块编写的程序必然在Web服务上,选项A错误。第2行的路由与第3行的视图函数定义了首页的内容,访问URL为http://192.168.1.100:5000/,第6行的路由与第7行的视图函数定义了接收数据、处理数据、存储数据的过程。选项B的URL中缺少端口号。智能终端通过IOT模块连到Wi-Fi再向服务器发送数据,Wi-Fi信号一般需要路由器发射,因此选项C正确。
	\end{enumerate}


\end{enumerate}


\newpage
