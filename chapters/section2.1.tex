\chapter{必修2 $\cdot$ 信息系统与社会}\index{信息系统与社会}
\section{信息系统概述}\index{信息系统概述}

\begin{enumerate}
%% ============= 1
\item 答案:C。注意是先互联网再数据——现在是数据时代,大数据对社会经济发展有着无可估量的作用。

%% ============= 2
\item 答案:C。

%% ============= 3
\item 答案:D。

%% ============= 4
\item 答案:C。

%% ============= 5
\item 答案:D。

%% ============= 6
\item 答案:B。

%% ============= 7
\item 答案:B。

%% ============= 8
\item 答案:A。信息系统的局限性有:外部环境的依赖,本身的安全隐患,技术鸿沟。目前看,对电力的依赖是无法消除的;自身的完全性可以很大程度上消除,无法完全消除。

%% ============= 9
\item 答案:C。选项A:信息社会特征:信息经济、网络社会、在线政府和数字生活。选项D:信息社会指数ISI在0至1之间,值越高表明信息社会发展水平越高。

%% ============= 10
\item 考查信息系统的组成,人工智能应用。
	\begin{enumerate}[label=$(\arabic*)$]
	\item 硬件是信息系统中看得见摸得着的部分,有摄像机(这属于输入设备)、交换机、服务器、工作站、区间测速提示牌(这属于输出设备)、超速显示屏。
	\item 开发者、维护者、使用者都是信息系统用户,有程序设计员、司机、高速交警、设备操作与维护人员。
	\item 车票自动设别,它属于模式识别中的文字识别,该人工智能实现方法一般是通过深度学习训练而来联结主义方法。
	\end{enumerate}



\end{enumerate}


\newpage
