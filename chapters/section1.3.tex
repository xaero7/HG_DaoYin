\section{数据处理与应用}\index{数据处理与应用}

\begin{enumerate}
%% ============= 1
\item 答案:D。

%% ============= 2
\item 答案:C。如果使用选项A,那么第16行的结果也是3,而不是13。D项,在Excel中,区域表示法应该用冒号,若有逗号则只有两个单元格。

%% ============= 3
\item 答案:B。对于数据不符合的单元格要审核之后再删除,而第16行的数据应该修改而不是删除。

%% ============= 4
\item 答案:B。考查pandas数据处理drop()函数、groupby()函数功能的理解。注释如下:
\begin{lstlisting}[numbers=left]
import pandas as pd									# 导入并使用pd作为别名
df = pd.read_csv("mnxk.csv", sep=",")				# 读取数据
df1 = df.drop("已选科目数", axis=1)				    # 删除“已选科目数”列
print(df.head())									# 打印df的前5行
print(df1.head())									# 打印df1的前5行
sc=df1.groupby("班级", as_index=False).count()	    # 按“班级”分组
m = len(df)											# df的行数
n = len(df1)										# df1的行数
print(sc)											# 分组后的数据
\end{lstlisting}
注意pandas的很多操作处理后原始数据都不会改变。比如,第3行drop()函数删除了“已选科目数”列,参数“axis=1”指明了这是列而不是行。该函数调用后,产生了一个新的数据集合并赋值给对象df1,而原始的数据集合df未曾变化。选项B就考查了pandas数据处理的这个特点:第4行打印的结果是原始数据的前5行,包含“已选科目数”这列数据,而第5行的输出的5行数据虽然大部分与前面相同,但不含“已选科目数”这列数据。选项C考查的是df和df1数据对象的行数是否相同,由于没有删除行,行数必然是一样的。选项D,第6行的分组可以让相同班级的数据合并成一行数据,这个数据的每列数据是原先该列数据的非空单元格个个数(即count()函数的功能)。如,若原始数据如左侧所示,则执行第6行分组语句后的结果如右侧所示。在右侧数据中,“1班”的“Name”值是4,表示原始数据中1班“Name”列数据非空单元格个有4个;“1班”的“物理”值是2,表示原始数据中1班“物理”列数据非空单元格个有2个(相当于1班有两个2人选了物理)。
\begin{lstlisting}
   班级 Name  物理 历史 技术 化学          班级 Name 物理 历史 技术 化学
0  1班  张三丰   1   1   1                0  1班  4   2   1   2   2
1  2班   郭靖    1       1                1  2班  2   1   0   1   0
2  1班  小龙女   1           1            2  3班  2   2   1   1   2
3  2班  李秋水                
4  3班   杨过    1       1   1
5  1班  令狐冲                
6  3班  任我行   1   1       1
7  1班   黄蓉            1   1
\end{lstlisting}

%% ============= 5
\item 答案:D。图1-3-4是典型的分布式批处理工作流程图,是对静态数据的处理,选项A错。选项B:Hadoop是分布式批处理处理框架,相比流数据处理框剪肯定要慢,况且现如今广告词都不能用最了,你还会选它么?图示很明显,分布式处理必然需要分布式存储数据,不是集中存储在同一个节点上。

%% ============= 6
\item 答案:B。考点还是在静态数据处理和实时流数据处理的区分,该流程图是静态数据的分布式批处理流程图,不适合处理实时流数据。

%% ============= 7
\item 答案:B。航班飞行时间和已飞航线都是“已经产生的数据”,是静态数据,无需预测。

%% ============= 8
\item 答案:D。当前速度是实时在变的数据,属于流数据。降落地点是固定已知的,属于静态数据。

%% ============= 9
\item 答案:C。

%% ============= 10
\item 考查pandas数据处理与应用。
	\begin{enumerate}[label=$(\arabic*)$]
	\item 考查数据处理的实际用途,帮助理解题目情景。
	\item 考查pandas数据格式的识别。
\setcounter{qnumber}{1}
\begin{lstlisting}[numbers=left]
def s_review(c):    
    for r in range(df.shape[0]):            # 批阅1个单选题
        if df.at[r, qnum[c]] == `\clozeblank{2}`:
            tmp = 3
            df.at[r, score[c-2]] = tmp
            df.at[r, score[10]] += tmp      # 计算总分,存入"sum"列
qnum = df.columns
sans = "BDCABDDBCB"                         # 本次作业的标准答案
score=["sc1","sc2","sc3","sc4","sc5","sc6","sc7","sc8","sc9","sc10","sum"]
for c in score:
    df[c] = 0
for c in range(2,12):                       # 逐题批阅
    `\clozeblank{2}`
print(df)
df.to_excel("客观题成绩.xlsx", index=False)  # 保存结果
\end{lstlisting}
	\item 解题过程:
		\begin{enumerate}[label=$(\alph*)$]
		\item 从第7行主程序开始阅读,对pandas程序阅读,一定要直到变量保存的数据是什么?数据的结构是怎样的?
		\item 第7行qnum保存了数据对象df的所有列名称,即\lstinline|["name", "snum", "ans1", "ans2", ..., "ans10"]|。第10行的循环,结合第9行score列表中的数据可以知道第11行在数据对象df中新增了很多数据列,列名称分别是“sc1”、“sc2”、“sc3”、……、“sum”,每列的值都是0。这也是pandas的特点,数据列直接可以参与算术运算、关系运算和赋值操作,每种操作都可以将该列的所有行都进行相应处理。
		\item 第②空处所进行的循环是逐题批阅。原始表格数据中一题就是一列数据,列序号是$2 \sim 12$,刚好能对上这里的循环范围。因此,第12行的循环变量$c$相当于列序号——不过,pandas需要的是列名称,这就需要qnum中对应的列名称来引用原始数据了。这里需要调用第一行的s\_review()函数。
		\item 阅读s\_review()函数。第2行df.shape可以返回数据对象df的维度“形状”:行数(df.shape[0])和列数(df.shape[1]),因此$r$就是行索引号。由\lstinline|df.at[r, qnum[c]]|操作可知qnum[c]必然是列名称,结合前面的分析可以知道$c$必然是列序号。由于列表qnum中索引$2$号的列名称才是第一题名称“ans1”,因此$c$期望的值也应该从$2$开始。那么第②空的函数调用就好办了:函数名已知的,参数作用也推知了,所以答案应该是\lstinline|s_revieww(c)|,就让$c$的值从2开始传递、调用函数。另外从s\_review()函数的结构上看,它有return语句返回值,所以这空也无需考虑赋值——直接调用即可。
		\item 再回到第3行程序,\lstinline|df.at[r, qnum[c]]|取得了$c$列每个人填写的答案,它们需要与标准答案做比较,标准答案保存再sans字符串中,它的索引号是从0开始的,所以第①空的答案是\lstinline|sans[c-2]|。
		\item 第5行的程序是将该行(第$r$行)对应的得分列赋值为tmp分分值(如“ans1”列对应的分值是“sc1”列)。第6行的程序是将该分值累加到它的总分中去(即“sum”列,它的值是10个选择题的得分累加而来)。
		\end{enumerate}
	\end{enumerate}


%% ============= 11
\item 考查文本数据处理、分词、字符串统计与字典的应用。
	\begin{enumerate}[label=$(\arabic*)$]
\setcounter{qnumber}{1}
\begin{lstlisting}[numbers=left]
import jieba                                     # 导入jieba模块
import pandas as pd
text = open("news.txt", encoding="utf-8").read() # 打开文本文件
words = jieba.lcut(text, cut_all=False)          # 分词
counts = {}
for name in words:
    if len(name) != 1 and not ("a" < name[0] < "z") and not ("0" < name[0] < "9"):
        if name in counts:
            counts[name] += 1                    # 词语已出现过
        else:
            counts[name] = 1                     # 词语第一次出现
# 字典转化为DataFrame格式存储
df = pd.DataFrame(list(counts.items()), columns=["词", "次数"])
df = df.sort_values("次数", ascending=False)     # 按“次数”降序排序
print(df)
\end{lstlisting}
	\item jieba是目前常用的分词模块,它是一个基于词典分词的模块。模块导入后,程序再第3行通过python的内置open()函数打开了文本文件,read()函数可以读取文件中的所有数据。第4行调用了jieba的lcut()函数进行分词,函数名中的“l”表示分词结果数据是一个列表(即list,这里了解即可,无需记忆),函数的cut\_all参数设定为False表示是精准分词,不会分隔“词中词”,当该参数设定为True时表示全模式分词,会分隔所有词。如“中华人民共和国”,False模式下结果是一个词\lstinline|["中华人民共和国"]|,True模式下会有多个词\lstinline|["中华", "人民", "共和国", "中华人民共和国"]|。对于words列表中的每个词,第7行程序过滤掉了单字、字母开头的、数字开头的字符串,因此答案选C。
	\item 本小题考查jieba分词的规则特点,因为它是用现有的词典进行分词的,因此想要添加一个新词时,只需在分词前添加该词再进行分词即可。具体可以通过\lstinline|jieba.add_word( "公益活动")|来添加该词。
	\item 第8行程序先判定某个单词name是否在字典counts的键名中出现过,如果出现过,则直接根据该键名取出其键值,然后加1后仍然存放在该键名上。else分支就是该键名第一次出现,该键值初始为1,答案是\lstinline|counts[name] = 1|。
	\end{enumerate}


%% ============= 12
\item 考查pandas数据处理与应用,matplotlib数据可视化。给出三行数据示例如下:

	\begin{enumerate}[label=$(\arabic*)$]
\setcounter{qnumber}{1}
\begin{lstlisting}
109, 2007-02-20 00:07:10, 121.443100, 31.273000,  0, 45, 0
109, 2007-02-20 00:08:06, 121.447600, 31.272000,  6, 22, 1
109, 2007-02-20 00:09:07, 121.452500, 31.271000, 46, 67, 1
\end{lstlisting}
\begin{lstlisting}[numbers=left]
import pandas as pd
import matplotlib.pyplot as plt             # 导入matplotlib模块
plt.rcParams["font.sans-serif"] = ["KaiTi"] # 图表中中文以楷体显示
df = pd.read_csv("Taxi_105.txt", sep=",")
df.columns = ["xh", "sk", "jd", "wd", "sd", "jj", "zkzt"]
df = `\underline{\hspace*{4cm}}`
\end{lstlisting}
	\item 第3行程序设置字体以便显示中文,这行代码了解即可。第4行程序读取了csv文件并转成DataFrame数据对象保存再df中。由于原始数据中没有标题行,程序的第5行指定了数据各列的标题。从处理结果上看,“速度”、“夹角”列都可以删除,因此删除这两列数据都可以。drop()函数删除列的语法是\lstinline|df.drop("jj", axis=1)|,其中axis参数为1指明了删除的是列。
%\setcounter{qnumber}{1}
\begin{lstlisting}[numbers=left, firstnumber=7]
def pickup(rid):
    # 计算每次上客的停车时长,代码略
    return t, id1 	# 返回停车时长、上客停车时的数据行索引
# 计算该出租车当日载客次数
ty = []; px = []; py = []; count = 0
n = len(df)
for row in range(n-1):
    if df.at[row, "zkzt"]==0 and df.at[row+1,"zkzt"]==1: 	# 上客
        pt, id = pickup(row)
        ty.append(pt) 
        `\clozeblank{2}`
        px.append(df.at[id, "jd"])
        py.append(df.at[id, "wd"])        
print("该出租车当日载客次数为:", count)
# 上客平均时长四舍五入取整
print("上客平均时长、最大及最小时长(单位:秒):", `\clozeblank{2}`, max(ty), min(ty))
\end{lstlisting}
	\item 第12行程序中$n$取得了df数据对象的行数,第13行循环了$n-1$次,可以确定row就是数据对象df的行索引值。第14行的判定是说当前行载客状态是空,但是下一行的载客状态是有客则表示当前正在上客。由pickup()函数的注释和return语句可知,第15行中的变量pt保存了函数的第一个返回值停车时长,id保存了第二个值上客时的数据行索引。因此下面几行程序也容易理解:列表变量ty保存了各次上客前停车时长,列表px保存了各次上车时的经度值,py保存了各次上车时的纬度值。第①空似乎不需要填什么程序。不过从循环结束后的第20行程序上看,这里输出了count的值,为载客次数,再考查它的初值是零,因此需要在循环内进行次数统计,于是第①空答案是\lstinline|count += 1|。第②空是输出上客平均时长,上车次数是count,这就需要算出上车总时长(接到各个客人时需要空车停留、闲逛的总时长)。这里已经将各次上车时间保存在列表ty中,求和只需调用sum()函数即可,四舍五入保留整数可以使用round()函数,于是答案是\lstinline|round(suum(ty)/count)|。注意很多同学的答案写成了mean(ty)。在python内置函数中,只有求和sum()函数,求最值max()/min()函数,没有求平均值mean()函数。在pandas中,数据框对象有mean()方法,但它的格式是df.mean(),而且会求出df每一列的平均值,结果不是一个整数而是一个Series。这里不能用该函数。
\setcounter{qnumber}{1}
\begin{lstlisting}[numbers=left]
tx = range(count)
plt.figure(1)
plt.title("当日上客时长对比图(单位:秒)")
`\clozeblank{2}`   # 绘制图表分析当日搭载乘客的上客时长
plt.figure(2)
plt.title("当日上客地点分布图")
`\clozeblank{2}`   # 绘制图表分析当日搭载乘客的上客地点分布
plt.show()
\end{lstlisting}
	\item plt数据可视化最关键的三要素是:图表类型、横纵轴数据。分析左图可以确定是柱形图,那么使用\lstinline|plt.bar()|函数。纵轴数据从纵坐标的值与图表标题上可以看出是上客时长,数据应该是列表ty中的值。这里的横轴数据不是很明显,图中也未给出坐标轴标签,不过看数据还是可以知道就是一些序号,结合第一行的程序,还是可以确定横轴数据就是列表tx中的值,于是答案是\lstinline|plt.bar(tx, ty)|。同样的方法可以确定第②空的答案是\lstinline|plt.scatter(px, py)|。
	\end{enumerate}











\end{enumerate}


\newpage
