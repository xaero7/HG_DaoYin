\section{数据处理与应用}\index{数据处理与应用}

\begin{enumerate}
%% ============= 1
\item 答案:D。考查数据整理方法与目的。

%% ============= 2
\item 答案:C。
	\begin{enumerate}[label=\Alph*.]
	\item 实验误差是测量值和真实值之间的偏差,不是虚假信息。
	\item 正确。
	\item 同一个信息对于不同的人价值可能不一样。
	\item 信息是信号、消息中所包含的含义,必须依附与数字、文字、图形、图像等载体。
	\end{enumerate}

%% ============= 3
\item 答案:B。
	\begin{enumerate}[label=\Alph*.]
	\item 互联网上只有已数字化的信息,没有数字化当然查不到。
	\item 知识的获得是人利用自身已有的知识对信息进行加工,进而将新的信息纳入自己的知识结构的过程。检索到也只是看到,并不一定已内化成自己的知识。
	\item 天才也要记单词啊。
	\item 正确。
	\end{enumerate}

%% ============= 4
\item 答案:B。考查pandas数据处理drop()函数、groupby()函数功能的理解。注释如下:
\begin{lstlisting}[numbers=left]
import pandas as pd									# 导入并使用pd作为别名
df = pd.read_csv("mnxk.csv", sep=",")				# 读取数据
df1 = df.drop("已选科目数", axis=1)				    # 删除“已选科目数”列
print(df.head())									# 打印df的前5行
print(df1.head())									# 打印df1的前5行
sc=df1.groupby("班级", as_index=False).count()	    # 按“班级”分组
m = len(df)											# df的行数
n = len(df1)										# df1的行数
print(sc)											# 分组后的数据
\end{lstlisting}
注意pandas的很多操作处理后原始数据都不会改变。比如,第3行drop()函数删除了“已选科目数”列,参数“axis=1”指明了这是列而不是行。该函数调用后,产生了一个新的数据集合并赋值给对象df1,而原始的数据集合df未曾变化。选项B就考查了pandas数据处理的这个特点:第4行打印的结果是原始数据的前5行,包含“已选科目数”这列数据,而第5行的输出的5行数据虽然大部分与前面相同,但不含“已选科目数”这列数据。选项C考查的是df和df1数据对象的行数是否相同,由于没有删除行,行数必然是一样的。选项D,第6行的分组可以让相同班级的数据合并成一行数据,这个数据的每列数据是原先该列数据的非空单元格个个数(即count()函数的功能)。如,若原始数据如左侧所示,则执行第6行分组语句后的结果如右侧所示。在右侧数据中,“1班”的“Name”值是4,表示原始数据中1班“Name”列数据非空单元格个有4个;“1班”的“物理”值是2,表示原始数据中1班“物理”列数据非空单元格个有2个(相当于1班有两个2人选了物理)。
\begin{lstlisting}
   班级 Name  物理 历史 技术 化学          班级 Name 物理 历史 技术 化学
0  1班  张三丰   1   1   1                0  1班  4   2   1   2   2
1  2班   郭靖    1       1                1  2班  2   1   0   1   0
2  1班  小龙女   1           1            2  3班  2   2   1   1   2
3  2班  李秋水                
4  3班   杨过    1       1   1
5  1班  令狐冲                
6  3班  任我行   1   1       1
7  1班   黄蓉            1   1
\end{lstlisting}

%% ============= 5
\item 答案:D。
	\begin{enumerate}[label=\Alph*.]
	\item 若化成十进制计算:$10H=16D, 10B = 2D, 16D+2D=18D$。
	\item $1AH + 2AH = 44H$,注意十六进制下$A + A$等于$14$。
	\item 正确。
	\item 若化成十六进制计算:$10D + 10B = AH + 2H = CH$,即十六进制的值是$C$。
	\end{enumerate}

%% ============= 6
\item 答案:B。
	\begin{enumerate}[label=\Alph*.]
	\item 。
	\item 。
	\item 。
	\item 。
	\end{enumerate}

%% ============= 7
\item 答案:B。
	\begin{enumerate}[label=\Alph*.]
	\item 。
	\item 。
	\item 。
	\item 。
	\end{enumerate}

%% ============= 8
\item 答案:D。
	\begin{enumerate}[label=\Alph*.]
	\item 。
	\item 。
	\item 。
	\item 。
	\end{enumerate}

%% ============= 9
\item 答案:C。
	\begin{enumerate}[label=\Alph*.]
	\item 。
	\item 。
	\item 。
	\item 。
	\end{enumerate}

%% ============= 10
\item 答案:B。
	\begin{enumerate}[label=\Alph*.]
	\item 。
	\item 。
	\item 。
	\item 。
	\end{enumerate}

%% ============= 11
\item 考查信息编码、容量计算。
	\begin{enumerate}[label=$(\arabic*)$]
	\item {\kaishu 视频容量=每帧图像容量$\times$帧频,每帧图像容量=像素点数$\times$量化位数}。依题意,单张图像容量是:$\dfrac{1280 \times 720 \times 24}{8 \times 1024 \times 1024} \approx 2.64$MB。因此视频容量是:$2.64 \times 5 \times 60 \times 25 = 19800$MB。压缩比至少是$39.6:1$才能压缩到500MB以内。答案是$40:1$。
	\item 加入数据不会改变原来的压缩比,相当于不会改变原先的压缩编码方式(真因为如此,加入音频后的视频容量增加,压缩比不变的话,压缩之后的作品容量也增加,势必会超过500MB,因此必须重新设定新的、更大的压缩比才能压缩到500MB以内,这应该是本题想考查的一个实际应用情景)。
	\item 压缩会使画面不清晰,原因压缩比太大,或者压缩算法太差。改进办法是可以换一种压缩算法(换一个压缩软件),或者保证内容完整的前提下,减少画面尺寸、缩短时长等。
	\end{enumerate}

\end{enumerate}


\newpage
